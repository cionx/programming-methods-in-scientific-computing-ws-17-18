\section{}

Wir erweitern die Klasse \texttt{Matrix} um eine Methode zum Berechnen der Transponierten, um im Folgenden die Ergebnisse überprüfen zu können.

\lstinputlisting[style=pythoncode, firstline = 66, lastline = 71]{chapter_03/exercise_03_05.py}

Wir bestimmen die Cholesky-Matrix eintragsweise.

\lstinputlisting[style=pythoncode, firstline = 94, lastline = 110]{chapter_03/exercise_03_05.py}

Für die gegebene Matrix
\[
    A
  = \begin{pmatrix}
      1 & 2 &  1 \\
      2 & 5 &  2 \\
      1 & 2 & 10
    \end{pmatrix}
\]
erhalten wir das folgende Ergebnis:

\begin{consoleoutput}
>>> A = Matrix([[1,2,1],[2,5,2],[1,2,10]])
>>> L = cholesky(A)
>>> print(L)
[1.0 0   0  ]
[2.0 1.0 0  ]
[1.0 0.0 3.0]
>>> print(L * L.transpose())
[1.0 2.0 1.0 ]
[2.0 5.0 2.0 ]
[1.0 2.0 10.0]
\end{consoleoutput}

Für die gegebene Matrix
\[
    B
  = \begin{pmatrix}
      1.01 \cdot 10^{-2}  & 0.705 &  1.42 \cdot 10^{-2} \\
      0.705               & 49.5  &  1 \\
      1.42 \cdot 10^{-2}  & 1     &  1
    \end{pmatrix}
\]
erhalten wir das folgende Ergebnis:

\begin{consoleoutput}
>>> B = Matrix([[1.01E-2, 0.705, 1.42E-2],[0.705,49.5,1],[1.42E-2,1,1]])
>>> L = cholesky(B)
>>> print(L)
[0.1004987562112089  0                    0                 ]
[7.015012190980423   0.5381486415443629   0                 ]
[0.14129528100981847 0.016374437298272527 0.9898320672556135]
>>> print(L * L.transpose())
[0.010100000000000001 0.705 0.014200000000000003]
[0.705                49.5  1.0                 ]
[0.014200000000000003 1.0   1.0                 ]
\end{consoleoutput}
