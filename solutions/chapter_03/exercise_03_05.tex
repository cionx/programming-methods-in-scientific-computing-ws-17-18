\section{}

Wir bestimmen die Cholesky-Zerlegung eintragsweise.

\lstinputlisting[style=pythoncode, linerange={2-20}]{chapter_03/exercise_03_05.py}

Für die gegebenen Matrizen
\[
    A
  = \begin{pmatrix}
      1 & 2 &  1 \\
      2 & 5 &  2 \\
      1 & 2 & 10
    \end{pmatrix}
    \quad\text{und}\quad
    B
  = \begin{pmatrix}
      1.01 \cdot 10^{-2}  & 0.705 &  1.42 \cdot 10^{-2} \\
      0.705               & 49.5  &  1 \\
      1.42 \cdot 10^{-2}  & 1     &  1
    \end{pmatrix}
\]
testen wir unser Programm mit dem folgenden Code:

\lstinputlisting[style=pythoncode, linerange={22-24,31-33,40-41,48-50,57-59,66-67}]{chapter_03/exercise_03_05.py}

Wir erhalten den folgenden Output:

\begin{consoleoutput}
$ python exercise_03_05.py
A:
[1 2 1 ]
[2 5 2 ]
[1 2 10]
L:
[1.0 0   0  ]
[2.0 1.0 0  ]
[1.0 0.0 3.0]
L * L^T:
[1.0 2.0 1.0 ]
[2.0 5.0 2.0 ]
[1.0 2.0 10.0]
B:
[0.0101 0.705 0.0142]
[0.705  49.5  1     ]
[0.0142 1     1     ]
L:
[0.1004987562112089  0                    0                 ]
[7.015012190980423   0.5381486415443629   0                 ]
[0.14129528100981847 0.016374437298272527 0.9898320672556135]
L * L^T:
[0.010100000000000001 0.705 0.014200000000000003]
[0.705                49.5  1.0                 ]
[0.014200000000000003 1.0   1.0                 ]
\end{consoleoutput}




