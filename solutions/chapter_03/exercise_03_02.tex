\section{}

Wir erweitern die gegeben Klasse \texttt{Polynomial} um eine Methode \texttt{derivative} zum Ableiten, sowie eine Methode \texttt{antiderivative} zum Bilden einer Stammfunktion.
Dabei wählen wir die \enquote{Integrationskonstante} als $0$.
Wir definieren außerdem Funktionen zum Ausgeben von Polynomen durch die \texttt{print}-Funktion.

\lstinputlisting[style=pythoncode, firstline=1, lastline=49]{chapter_03/exercise_03_02.py}

Für das gegebene Polynom $p(x) = 3 x^2 + 2 x + 1$ testen wir unser Programm mit dem folgenden Code:

\lstinputlisting[style=pythoncode, linerange={53-55, 60-61, 66-67,72-73}, ]{chapter_03/exercise_03_02.py}

Dabei erhalten wir den folgenden Output:

\begin{consoleoutput}
$ python exercise_03_02.py
The given polynmial p:
1 x^0 + 2 x^1 + 3 x^2
The derivative of p:
2 x^0 + 6 x^1
The antiderivative of p:
0 x^0 + 1.0 x^1 + 1.0 x^2 + 1.0 x^3
Taking antiderivative and then derivative:
1.0 x^0 + 2.0 x^1 + 3.0 x^2
\end{consoleoutput}




