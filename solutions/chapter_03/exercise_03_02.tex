\section{}

Wir weitern die gegeben Klasse \texttt{Polynomial} um Methoden zum Ableiten, sowie Bilden einer Stammfunktion.
Dabei wählen wir die \enquote{Integrationskonstante} als $0$.

\lstinputlisting[style=pythoncode, firstline=1, lastline=36]{chapter_03/exercise_03_02.py}

% \begin{pythonlisting}
% class Polynomial:
%     def __init__(self, coefficients):
%         self.coeff = coefficients
%         
%     def __call__(self, x):
%         s = 0
%         for i in range(len(self.coeff)):
%             s += self.coeff[i]*x**i
%         return s
%     
%     def __add__(self, other):
%         l = []
%         if len(self.coeff) > len(other.coeff):
%             l += self.coeff
%             for i in range(len(other.coeff)):
%                 l[i] += other.coeff[i]
%         else:
%             l += other.coeff
%             for i in range(len(self.coeff)):
%                 l[i] += self.coeff[i]
%         return Polynomial(l)
%     
%     def __eq__(self, other):
%         return self.coeff == other.coeff
%     
%     def derivative(self):
%         l = []
%         for i in range(1,len(self.coeff)):
%             l.append(i * self.coeff[i])
%         return Polynomial(l)
%     
%     def antiderivative(self):
%         l = [0]
%         for i in range(len(self.coeff)):
%             l.append(self.coeff[i]/(i+1))
%         return Polynomial(l)
% \end{pythonlisting}

Für das gegebene Polynom $p(x) = 3 x^2 + 2 x + 1$ erhalten wir die folgenden Ergebnisse:

\begin{consoleoutput}
    >>> p = Polynomial([1,2,3])
    >>> p.derivative().coeff
    [2, 6]
    >>> p.antiderivative().coeff
    [0, 1.0, 1.0, 1.0]
    >>> p.antiderivative().derivative().coeff
    [1.0, 2.0, 3.0]
\end{consoleoutput}




