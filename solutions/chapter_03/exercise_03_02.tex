\section{}

Wir erweitern die gegeben Klasse \texttt{Polynomial} um eine Methode \texttt{derivative} zum Ableiten, sowie eine Methode \texttt{antiderivative} zum Bilden einer Stammfunktion.
Dabei wählen wir die \enquote{Integrationskonstante} als $0$.
Wir definieren außerdem Funktionen zum Ausgeben von Polynomen durch die \texttt{print}-Funktion.

\lstinputlisting[style=pythoncode, firstline=1, lastline=79]{chapter_03/exercise_03_02.py}

Für das gegebene Polynom $p(x) = 3 x^2 + 2 x + 1$ testen wir unser Programm mit dem folgenden Code:

\lstinputlisting[style=pythoncode, linerange={85-88, 92-95, 99-102, 106-108}, ]{chapter_03/exercise_03_02.py}

Dabei erhalten wir den folgenden Output:

\begin{consoleoutput}
The given polynmial p:
3x^2 + 2x + 1
The derivative of p:
6x + 2
The antiderivative of p:
1.0x^3 + 1.0x^2 + 1.0x
Taking antiderivative and then derivative:
3.0x^2 + 2.0x + 1.0
\end{consoleoutput}




