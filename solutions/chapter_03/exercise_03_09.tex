\section{}
\label{section: better trapeze}





\subsection{}

Wir definieren eine neue Funktion \texttt{powertrapeze}, welche das angegebene Verfahren implementiert:

\lstinputlisting[style=pythoncode, firstline = 9, lastline = 21]{chapter_03/trapeze.py}

Hiermit berechnen die Approximationen für $\int_0^\pi \sin(x) \,\text{d}x$ für $m = 1, \dotsc, 10$ mit dem folgenden Code:

\lstinputlisting[style=pythoncode, firstline = 5, lastline = 11]{chapter_03/exercise_03_09.py}

Wir erhalten den folgenden Output:
\begin{consoleoutput}
Calculate trapeze estimate for int. of sin from 0 to pi, 2^m intervals:
 m      estimate                error
 1      1.5707963267948966      0.42920367320510344200
 2      1.8961188979370398      0.10388110206296019555
 3      1.9742316019455508      0.02576839805444919307
 4      1.9935703437723395      0.00642965622766045186
 5      1.9983933609701445      0.00160663902985547224
 6      1.9995983886400386      0.00040161135996141795
 7      1.9998996001842035      0.00010039981579645918
 8      1.9999749002350518      0.00002509976494824429
 9      1.9999937250705773      0.00000627492942273378
10      1.9999984312683816      0.00000156873161838433
\end{consoleoutput}





\subsection{}

Es fällt auf, dass sich der Fehler in jedem Schritt etwa geviertelt wird.
Bezeichnet $a_n$ die $n$-te Approximation, so gilt $a_0 \leq a_1 \leq \dotsb \leq a_n$, da $\sin$ auf $[0,\pi]$ konkav ist.
Deshalb ist die Vermutung äquivalent dazu, dass die Quotienten $(a_i - a_{i+1})/(a_{i+1} - a_{i+2})$ ungefähr $4$ sind.
Dies testen wir mit dem folgenden weiteren Code:

\lstinputlisting[style=pythoncode, firstline = 32, lastline = 35]{chapter_03/exercise_03_09.py}

Wir erhalten den folgenden Output:

\begin{consoleoutput}
Quotients of any two subsequent differences of estimates:
4.164784400584785
4.039182316416593
4.009677144752887
4.002411992937073
4.00060254408483
4.000150607761501
4.000037649528035
4.000009414842847
\end{consoleoutput}

Es fällt auf, dass das Verhältnis sogar gegen $4$ zu gehen scheint.





\subsection{}

Wir berechnen die Approximationen für $\int_0^2 3^{3x-1} \,\text{d}x$ für $m = 1, \dotsc, 10$ mit dem folgenden Code:

\lstinputlisting[style=pythoncode, firstline = 53, lastline = 59]{chapter_03/exercise_03_09.py}

Wir erhalten den folgenden Output:

\begin{consoleoutput}
Calculate trapeze estimate for int. of 3^(3x-1) from 0 to 2, 2^m intervals:
 m      estimate
 1      130.66666666666665719276
 2       89.58204463929762084717
 3       77.74742639121230070032
 4       74.66669853961546721166
 5       73.88840395800384897029
 6       73.69331521665949935596
 7       73.64451070980437918934
 8       73.63230756098684537392
 9       73.62925664736960129630
10       73.62849391106399821183
\end{consoleoutput}


Da $f$ konvex ist, sind die Approximationen $b_n$ monoton fallend.
Die Vermutung lässt sich erneut durch das Betrachten der Quotienten $(b_i - b_{i+1})/(b_{i+1} - b_{i+2})$ überprüfen.
Hierfür nutzen wir (erneut) den folgenden Code:

\lstinputlisting[style=pythoncode, firstline = 74, lastline = 77]{chapter_03/exercise_03_09.py}

Wir erhalten den folgenden Output:

\begin{consoleoutput}
Quotients of any two subsequent differences of estimates:
3.471562932248868
3.841500716121706
3.958305665211694
3.9894387356667425
3.9973509398114637
3.9993371862347957
3.9998342622879792
3.999958563440565
\end{consoleoutput}

Unsere Vermutung scheint sich zu bestätigen.
