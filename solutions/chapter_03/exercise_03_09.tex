\section{}
\label{section: better trapeze}





\subsection{}

Wir definieren eine neue Funktion \texttt{Trapeze}:

\lstinputlisting[style=pythoncode, firstline = 3, lastline = 15]{chapter_03/exercise_03_09.py}

Wir berechnen die Werte für $m = 1, \dotsc, 10$, also für $m_{\text{max}} = 10$, und geben diese, zusammen mit der jeweiligen Abweichung vom exakten Ergebnis, in Tabellenform aus:

\begin{consoleoutput}
>>> from math import sin, pi
>>> m = 10
>>> results = trapeze(sin, 0, pi, m)
>>> for i in range(m):
...     print("m={:2d}\t{}\t{:.20f}".format(i+1, results[i], 2-results[i]))
... 
m= 1    1.5707963267948966      0.42920367320510344200
m= 2    1.8961188979370398      0.10388110206296019555
m= 3    1.9742316019455508      0.02576839805444919307
m= 4    1.9935703437723395      0.00642965622766045186
m= 5    1.9983933609701445      0.00160663902985547224
m= 6    1.9995983886400386      0.00040161135996141795
m= 7    1.9998996001842035      0.00010039981579645918
m= 8    1.9999749002350518      0.00002509976494824429
m= 9    1.9999937250705773      0.00000627492942273378
m=10    1.9999984312683816      0.00000156873161838433
\end{consoleoutput}




\subsection{}

Es fällt auf, dass sich der Fehler in jedem Schritt etwa geviertelt wird.
Bezeichnet $a_n$ die $n$-te Approximation, so gilt $a_0 \leq a_1 \leq \dotsb \leq a_n$; deshalb ist diese Beobachtung äquivalent dazu, dass die Quotienten $(a_i - a_{i+1})/(a_{i+1} - a_{i+2})$ ungefähr $4$ sind.

\begin{consoleoutput}
>>> for i in range(m-2):
...     print( (results[i] - results[i+1])/(results[i+1] - results[i+2]) )
... 
4.164784400584785
4.039182316416593
4.009677144752887
4.002411992937073
4.00060254408483
4.000150607761501
4.000037649528035
4.000009414842847
\end{consoleoutput}


\subsection{}

Wir erhalten die folgenden Approximationen:

\begin{consoleoutput}
>>> m = 10
>>> f = (lambda x : 3**(3*x-1))
>>> results = trapeze( f, 0, 2, m)
>>> for i in range(m):
...     print("m={:2d}\t{:24.20f}".format(i+1, results[i]))
... 
m= 1    130.66666666666665719276
m= 2     89.58204463929762084717
m= 3     77.74742639121230070032
m= 4     74.66669853961546721166
m= 5     73.88840395800384897029
m= 6     73.69331521665949935596
m= 7     73.64451070980437918934
m= 8     73.63230756098684537392
m= 9     73.62925664736960129630
m=10     73.62849391106399821183
\end{consoleoutput}

Da $f$ konvex ist, sind die Approximationen $b_n$ monoton fallend.
Die Vermutung lässt sich erneut durch das Betrachten der Quotienten $(b_i - b_{i+1})/(b_{i+1} - b_{i+2})$ überprüfen:

\begin{consoleoutput}
>>> for i in range(m-2):
...     (results[i] - results[i+1])/(results[i+1] - results[i+2])
... 
3.471562932248868
3.841500716121706
3.958305665211694
3.9894387356667425
3.9973509398114637
3.9993371862347957
3.9998342622879792
3.999958563440565
\end{consoleoutput}

Unsere Vermutung scheint sich zu bestätigen.
