\section{}

Mithilfe elementarer Zeilenumformungen, die in der Klasse \texttt{Matrix} implementiert sind, lässt sich nun der Gauß-Algorithmus zum Invertieren von Matrizen implementieren.

\lstinputlisting[style=pythoncode, firstline = 1, lastline = 35]{chapter_03/exercise_03_06.py}

Wir testen unser Programm anhand der gegebenen Matrix
\[
  A
  = \begin{pmatrix*}[r]
       3  & -1  &  2  \\
      -3  &  4  & -1  \\
      -6  &  5  & -2
    \end{pmatrix*}
\]
mit dem folgenden Code:

\lstinputlisting[style=pythoncode, linerange={39-42,48-54,64-66}]{chapter_03/exercise_03_06.py}

Dabei nutzen wir erneut die Klasse \texttt{Rational}, um ein genaues Rechnen zu erlauben.
Wir erhalten den folgenden Output:

\begin{consoleoutput}
A:
[3  -1 2 ]
[-3 4  -1]
[-6 5  -2]
A^(-1) with rationals:
[-1162261467/3486784401 3099363912/3486784401 -2711943423/3486784401]
[0/43046721             28697814/43046721     -14348907/43046721    ]
[59049/59049            -59049/59049          59049/59049           ]
A^(-1) with floats:
[-0.3333333333333333 0.8888888888888888 -0.7777777777777778]
[0.0                 0.6666666666666666 -0.3333333333333333]
[1.0                 -1.0               1.0                ]
Checking if A*B == I (using rationals):
True
\end{consoleoutput}
