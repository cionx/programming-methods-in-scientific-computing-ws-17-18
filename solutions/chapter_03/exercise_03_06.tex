\section{}

Wir nutzen die Klassen \texttt{Rational} und \texttt{Matrix} aus \ref{section: LU decomposition}, sowie die dort definierten Hilfsfunktionen.
Zudem erweitern wir die Klasse \texttt{Matrix} um zwei weitere Methoden zum Durchführen elementarer Zeilenumformungen.

\lstinputlisting[style=pythoncode, firstline = 120, lastline = 129]{chapter_03/exercise_03_06.py}

Mithilfe elementarer Zeilenumformungen lässt sich nun der Gauß-Algorithmus zum Invertieren von Matrizen implementieren.

\lstinputlisting[style=pythoncode, firstline = 159, lastline = 188]{chapter_03/exercise_03_06.py}

Für die gegebene Matrix
\[
  A
  = \begin{pmatrix*}[r]
       3  & -1  &  2  \\
      -3  &  4  & -1  \\
      -6  &  5  & -2
    \end{pmatrix*}
\]
erhalten wir das folgende Ergebnis:
\begin{consoleoutput}
>>> A = Matrix([[3,-1,2],[-3,4,-1],[-6,5,-2]])
>>> B = invert(A)
>>> A = A.mapentries(Rational)
>>> print(A*B == identitymatrix(3))
True
>>> print(B.mapentries(float))
[-0.3333333333333333 0.8888888888888888 -0.7777777777777778]
[0.0                 0.6666666666666666 -0.3333333333333333]
[1.0                 -1.0               1.0                ]
\end{consoleoutput}
