\section{}

Ist $T_n(x) = \sum_{k=0}^n x^k/k!$ das $k$-te Taylorpolynom für $f$ an der Entwicklungsstelle $0$, so gilt für das Restglied $R_n(x) \coloneqq e^x - T_n(x)$, dass es für jedes $x \in \Real$ ein $\xi$ zwschen $0$ und $x$ gibt, so dass
\[
    R_n(x)
  = \frac{f^{(n+1)}(\xi)}{(n+1)!} \xi^n
  = \frac{e^\xi \xi^n}{(n+1)^!}\,.
\]
Für alle $x \geq 0$ gilt $e^\xi \leq e^x \leq 3^x$, und somit gilt
\[
        \abs{R_n(x)}
  \leq  \frac{3^x x^n}{(n+1)^!}
  \qquad
  \text{für alle $x \geq 0$}.
\]
Für alle $x \leq 0$ gilt $e^\xi \leq e^0 = 1$, und somit
\[
        \abs{R_n(x)}
  \geq  \frac{(-x)^n}{(n+1)^!}.
\]
Dies führt zu dem folgenden Code:

\lstinputlisting[style=pythoncode, firstline = 1, lastline = 16]{chapter_03/exercise_03_11.py}

Für etwa $x \geq 23$ und $x \leq -26$ führt funktioniert diese Approximation allerdings nicht mehr mit der gewünschten Genauigkeit, da die aufzuaddierenden Summanden $x^n/n!$ dann betragsmäßig zu klein werden.
