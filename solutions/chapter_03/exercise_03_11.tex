\section{}

Ist $T_n(x) = \sum_{k=0}^n x^k/k!$ das $k$-te Taylorpolynom für $f(x) = e^x$ an der Entwicklungsstelle $0$, so gilt für das Restglied $R_n(x) \coloneqq e^x - T_n(x)$, dass es für jedes $x \in \Real$ ein $\xi$ zwschen $0$ und $x$ gibt, so dass
\[
    R_n(x)
  = \frac{f^{(n+1)}(\xi)}{(n+1)!} \xi^n
  = \frac{e^\xi \xi^n}{(n+1)^!}\,.
\]
Für alle $x \geq 0$ gilt $e^\xi \leq e^x \leq 3^x$, und somit gilt
\[
        \abs{R_n(x)}
  \leq  \frac{3^x x^n}{(n+1)^!}
  \qquad
  \text{für alle $x \geq 0$}\,.
\]
Für alle $x \leq 0$ gilt $e^\xi \leq e^0 = 1$, und somit
\[
        \abs{R_n(x)}
  \leq  \frac{(-x)^n}{(n+1)^!}\,.
\]
Dies führt zu dem folgenden Code:

\lstinputlisting[style=pythoncode, firstline = 3, lastline = 18]{chapter_03/exercise_03_11.py}

Wir testen die Genauigkeit des Programms mit dem folgenden Code:

\lstinputlisting[style=pythoncode, firstline = 22, lastline = 28]{chapter_03/exercise_03_11.py}

Wir erhalten den folgenden (gekürzten) Output:
\begin{consoleoutput}
Comparison of exp_approx(x) and exp(x) up to 7 digits.
  x            approximation                    exact   difference (10 digits)
-30               -0.0000855                0.0000000    0.0000855145
-29                0.0000551                0.0000000   -0.0000550745
-28                0.0000050                0.0000000   -0.0000050079
-27               -0.0000045                0.0000000    0.0000044619
-26               -0.0000014                0.0000000    0.0000013633
-25               -0.0000006                0.0000000    0.0000006464
-24               -0.0000003                0.0000000    0.0000002671
-23               -0.0000000                0.0000000    0.0000000403
-22               -0.0000000                0.0000000    0.0000000071
-21               -0.0000000                0.0000000    0.0000000192
[...]
 21       1318815734.4832141       1318815734.4832146    0.0000004768
 22       3584912846.1315928       3584912846.1315918   -0.0000009537
 23       9744803446.2489052       9744803446.2489033   -0.0000019073
 24      26489122129.8434715      26489122129.8434715    0.0000000000
 25      72004899337.3858795      72004899337.3858795    0.0000000000
 26     195729609428.8387451     195729609428.8387756    0.0000305176
 27     532048240601.7988281     532048240601.7986450   -0.0001831055
 28    1446257064291.4738770    1446257064291.4750977    0.0012207031
 29    3931334297144.0424805    3931334297144.0419922   -0.0004882812
 30   10686474581524.4667969   10686474581524.4628906   -0.0039062500
\end{consoleoutput}

Für etwa $x \geq 23$ und $x \leq -26$ hat unsere Approximation nicht mehr die gewünschten Genauigkeit, da die aufzuaddierenden Summanden $x^n/n!$ dann zu klein werden.




