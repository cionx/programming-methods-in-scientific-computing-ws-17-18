\section{}

Wir definieren direkt Klasse \texttt{Matrix}, die über alle Methoden verfügt, die wir in diesem und den späteren Aufgabenteilen nutzen werden.

\lstinputlisting[style=pythoncode, firstline=3, lastline = 123]{chapter_03/matrices.py}

Wir definieren zudem Hilfsfunktionen, die wir im Weiteren nutzen werden:

\lstinputlisting[style=pythoncode, firstline=131]{chapter_03/matrices.py}

Die Assoziativität der Matrixmultiplikation testen wir mit dem folgenden Code:

\lstinputlisting[style=pythoncode, linerange={3-6,12-16,21-25,32-34}]{chapter_03/exercise_03_03.py}

Wir erhalten wir den folgenden Output:

\begin{consoleoutput}
A:
[0 1]
[1 0]
[1 1]
B:
[1 2 3 4]
[5 6 7 8]
C:
[1 0]
[0 1]
[1 0]
[0 1]
Checking if A(BC) == (AB)C:
True
\end{consoleoutput}



