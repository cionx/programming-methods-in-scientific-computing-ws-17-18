\section{}

Für die Funktion $f(x) = e^{x^2}$ gilt $f''(x) = (4x^2 + 2) e^{x^2}$.
Da $f''(x) > 0$ auf $[0,1]$ monoton steigend ist, gilt für alle $0 \leq a \leq b \leq 1$, dass
\[
        \abs{ \operatorname{E}(f,a,b) }
  \leq  \frac{(b-a)^3}{12} \max_{a \leq x \leq b} \abs{f''(x)}
  \leq  \frac{(b-a)^3}{12} f''(b)\,.
\]
Für alle $n \geq 1$ und $0 \leq k \leq n-1$ gilt deshalb
\[
        \abs*{ E\left( f, \frac{k}{n}, \frac{k+1}{n} \right) }
  \leq  \frac{1}{12 n^3} \left( 4 \left(\frac{k+1}{n}\right)^2 + 2 \right) \underbrace{e^{((k+1)/n)^2}}_{\leq e \leq 4}
  \leq  \frac{1}{12 n^3} (4 + 2) \cdot 4
  \leq  \frac{2}{n^3}\,.
\]
Der gesamte Fehler für eine Unterteilung von $[0,1]$ in $n$ Intervalle lässt sich deshalb ingesamt durch
\[
      n \cdot \frac{2}{n^3}
    = \frac{2}{n^2}
\]
abschätzen.
Dabei gilt
\begin{align*}
              \frac{2}{n^2} < 10^{-6}
  \iff        n^2 > 2 \cdot 10^6
  \iff        n > \sqrt{2} \cdot 10^3
  \impliedby  n > 1500\,.
\end{align*}
Für das verbesserte Trapezverfahren aus \ref{section: better trapeze} gilt mit $n = 2^m$, dass $n > 1500$ für $m \geq 11$.
Wir nutzen nun den folgenden Code, um die entsprechenden Approximationen für $m = 1, \dotsc, 11$ zu bestimmen:

\lstinputlisting[style=pythoncode, firstline = 5, lastline = 11]{chapter_03/exercise_03_10.py}

Wir erhalten den folgenden Output:

\begin{consoleoutput}
$ python exercise_03_10.py
Calculate trapeze estimate for int. of e^(x^2) from 0 to 1, 2^m intervals:
m= 1      1.57158316545863208091
m= 2      1.49067886169885532865
m= 3      1.46971227642966528748
m= 4      1.46442031014948170764
m= 5      1.46309410260642858148
m= 6      1.46276234857772702291
m= 7      1.46267939741858832292
m= 8      1.46265865883777390621
m= 9      1.46265347414312651964
m=10      1.46265217796637525538
m=11      1.46265185392199392744

\end{consoleoutput}
