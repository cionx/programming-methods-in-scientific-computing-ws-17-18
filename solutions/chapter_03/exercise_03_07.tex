\section{}

Wir Berechnen die QR-Zerlegung einer nicht-singulären Matrix $A$ durch Anwenden des Gram-Schmidt-Verfahrens auf die Spalten von $A$, von links nach rechts:

\lstinputlisting[style=pythoncode, firstline=2, lastline = 26]{chapter_03/exercise_03_07.py}

Für die gegebene Matrix
\[
    A
  = \begin{pmatrix*}[r]
      12 & -51  &   4 \\
       6 &  167 & -68 \\
      -4 &  24  & -41
    \end{pmatrix*}
\]
testen wir das Programm mithilfe des folgenden Codes:
  
\lstinputlisting[style=pythoncode, linerange={29-31, 38-40, 48-49, 56-57, 64-65}]{chapter_03/exercise_03_07.py}

Wir erhalten den folgenden Output:

\begin{consoleoutput}
$ python exercise_03_07.py
A:
[12 -51 4  ]
[6  167 -68]
[-4 24  -41]
Q:
[0.8571428571428571  -0.3942857142857143 -0.33142857142857124]
[0.42857142857142855 0.9028571428571428  0.03428571428571376 ]
[-0.2857142857142857 0.17142857142857143 -0.9428571428571428 ]
Q * Q^T:
[0.9999999999999998      1.474514954580286e-16 -1.6653345369377348e-16]
[1.474514954580286e-16   0.9999999999999998    4.996003610813204e-16  ]
[-1.6653345369377348e-16 4.996003610813204e-16 1.0                    ]
R:
[14.0 20.999999999999996 -14.000000000000002]
[0.0  175.0              -69.99999999999999 ]
[0.0  0.0                35.0               ]
Q*R:
[12.0 -51.0 4.000000000000002]
[6.0  167.0 -68.0            ]
[-4.0 24.0  -41.0            ]
\end{consoleoutput}

