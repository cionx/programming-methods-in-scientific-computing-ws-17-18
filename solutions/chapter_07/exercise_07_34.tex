\section{}

Wir nutzen eine selbstgeschriebene \texttt{Matrix}-Klasse mit der folgenden Header-Datei:

\lstinputlisting[style=cppcode]{chapter_07/matrix.hpp}

Der konkrete Code der Matrix-Klasse ist wie folgt gegeben:

\lstinputlisting[style=cppcode]{chapter_07/matrix.cpp}

Wir nutzen außerdem die \texttt{Polynom}-Klasse aus Aufgabe~37.
Für die Interpolation selbst nutzen wir nun das folgende Programm:

\lstinputlisting[style=cppcode]{chapter_07/exercise_07_34.cpp}

Wir erhalten damit in der Konsole den folgende Output:

\begin{consoleoutput}
$ g++ exercise_07_34.cpp polynomial.cpp matrix.cpp -o exercise_07_34 
$ ./exercise_07_34 
4.83486x^3 - 1.47748x
\end{consoleoutput}
