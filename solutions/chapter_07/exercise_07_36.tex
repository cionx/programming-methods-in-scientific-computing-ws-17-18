\section{}

Es sei $n = 5$ die Anzahl der Punkte und $h = 1/(n-1) = 1/4$ der Abstand der $n$ im Intervall $[0,1]$ gleichmäßig verteilten Punkte $x_i = ih$.
Aus der Randbedingung $u(0) = 0$ erhalten wir für die Werte $u_i = u(x_i)$, dass $u_0 = u(x_0) = u(0) = 0$.
Mit den Annäherungen
\begin{gather*}
          u''(x)
  \approx \frac{1}{h^2}( u(x+h) + u(x-h) - 2u(x))
\shortintertext{und}
          u'(x)
  \approx \frac{1}{h}( u(x+h) - u(x) )
\end{gather*}
erhalten wir aus der Differenzialgeichung $u''(x) - 5 u'(x) + 4u(x) = x^2$ für alle $i \geq 1$ die linearen Gleichungen
\begin{equation}
  \label{equation: discrete ode}
    \frac{1}{h^2}( u_{i+1} + u_{i-1} - 2 u_i )
  - \frac{5}{h}( u_{i+1} - u_i )
  + 4 u_i
  = x_i^2 \,.
\end{equation}
Um die Bedingunge $u'(1) = 0$ zu simulieren, fügen wir noch einen zusätzlichen Punkt $x_n = 1+h$ hinzu, und für den entsprechenden Funktionswert $u_n = u(1+h)$ die Gleichung
\begin{equation}
  \label{equation: approx right boundary condition}
  \frac{1}{h}( u_n - u_{n-1} ) = u'(1) = 0 \,.
\end{equation}
Wir erhalten somit ein lineares Gleichungssystem in den $n$ Unbekannten $u_1, \dotsc, u_n$, sowie $n-1$ Gleichungen der Form \eqref{equation: discrete ode} (für $i = 1, \dots, n-1$) und der zusätzlichen Gleichung \eqref{equation: approx right boundary condition}.
Um dieses zu Lösen, nutzen wir erneut unsere \texttt{Matrix}-Klasse.

Wir erhalten somit das folgende Programm:

\lstinputlisting[style=cppcode, firstline = 1, lastline = 63]{chapter_07/exercise_07_36.cpp}

Wir erhalten den folgenden Output:

\begin{consoleoutput}
x                  approx          exact
0.000000        | 0.000000      | -0.000000
0.250000        | -0.000451     | -0.010135
0.500000        | -0.014724     | -0.033187
0.750000        | -0.034856     | -0.068764
1.000000        | -0.129808     | -0.094135
\end{consoleoutput}
Unsere Annäherung ist nicht sonderlich exakt.
Dies lässt sich etwa damit erklären, dass wir in unserem Vorgehen die Bedingung $u'(1) = 0$ dadurch implementieren, dass $u_n = u_{n-1}$, also $u(1) = u(1+h) = u(5/4)$.
Die exakte Lösung verändert sich im Bereich $[1,5/4]$ allerdings verhältnismäßig stark, weshalb diese Umsetzung der Randbedingung $u'(1) = 0$ nicht sonderlich gut funktioniert.

Dieses Problem lässt nach, wenn mehr Punkte verwendet werden.
So enthält man etwa für $n = 24$ die folgenden Ergebnisse:
\begin{consoleoutput}
x                  approx          exact
0.000000        | 0.000000      | -0.000000
0.043478        | -0.000891     | -0.001071
0.086957        | -0.002016     | -0.002388
0.130435        | -0.003416     | -0.003989
0.173913        | -0.005131     | -0.005909
0.217391        | -0.007200     | -0.008181
0.260870        | -0.009659     | -0.010837
0.304348        | -0.012544     | -0.013902
0.347826        | -0.015886     | -0.017399
0.391304        | -0.019710     | -0.021343
0.434783        | -0.024036     | -0.025743
0.478261        | -0.028874     | -0.030595
0.521739        | -0.034226     | -0.035886
0.565217        | -0.040075     | -0.041586
0.608696        | -0.046391     | -0.047645
0.652174        | -0.053117     | -0.053993
0.695652        | -0.060172     | -0.060531
0.739130        | -0.067436     | -0.067126
0.782609        | -0.074747     | -0.073605
0.826087        | -0.081886     | -0.079745
0.869565        | -0.088570     | -0.085265
0.913043        | -0.094428     | -0.089813
0.956522        | -0.098987     | -0.092949
1.000000        | -0.101646     | -0.094135
\end{consoleoutput}
