\section{}

Wir nutzen erneut die bereits \texttt{Matrix}-Klasse.

Es sei $n = 5$ die Anzahl der Punkte und $h = 1/(n-1) = 1/4$ die Länge der Teilintervalle erhalten.
Aus den gebenen Randwerten erhalten wir für die Punkte $u_i = (ih)$, dass $u_0 = u(0) = 0$ und $u_n = u(1) = u(0)$.
Aus der Approximation
\[
          u''(x)
  \approx \frac{1}{h^2}( u(x+h) + u(x-h) - 2u(x))
\]
und der Bedingung $u'' = 4u$ die Gleichungen
\[
    4 u_i
  = \frac{1}{h^2}( u_{i+1} + u_{i-1} - 2u )
  \qquad
  \text{für alle $i = 1, \dotsc, n-1$} \,,
\]
also das lineare Gleichungssystem
\[
  \begin{pmatrix}
     C  & -1  &         &     &     \\
    -1  &  C  & \ddots  &     &     \\
        & -1  & \ddots  & -1  &     \\
        &     & \ddots  &  C  & -1  \\
        &     &         & -1  &  C
  \end{pmatrix}
  \begin{pmatrix}
    u_1     \\
    u_2     \\
    \vdots  \\
    u_{n-2} \\
    u_{n-1}
  \end{pmatrix}
  =
  \begin{pmatrix}
    u_0     \\
    0       \\
    \vdots  \\
    0       \\s
    u_n
  \end{pmatrix} \,,
\]
wobei $C = 2 + 4h^2$.

\lstinputlisting[style=cppcode, firstline = 1, lastline = 57]{chapter_07/exercise_07_35.cpp}

Wir erhalten in der Konsole den folgende Output:

\begin{consoleoutput}
$ g++ exercise_07_35.cpp matrix.cpp -o exercise_07_35
$ ./exercise_07_35                                  
x                 approximation   exact
0.00000000      | 0.00000000    | 0.00000000
0.25000000      | 1.05269418    | 1.04219061
0.50000000      | 2.36856190    | 2.35040239
0.75000000      | 4.27657010    | 4.25855891
1.00000000      | 7.25372082    | 7.25372082
\end{consoleoutput}


