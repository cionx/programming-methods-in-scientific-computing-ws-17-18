\section{}

Die Abbildung $f \colon \Real \to \Real$, $x \mapsto x^3 + 3x$ ist stetig mit $\lim_{x \to -\infty} f(x) = -\infty$ und $\lim_{x \to \infty} f(x) = \infty$;
nach dem Zwischenwertsatz ist $f$ deshalb surjektiv.
Außerdem ist $f$ differenzierbar mit $f'(x) = 3 x^2 + 3 > 0$ für alle $x \in \Real$, weshalb $f$ streng monoton steigend, und somit injektiv ist.
Die Abbildung $f$ ist also bijektiv, weshalb die Gleichung $f(x) = a$ für jedes $a \in \Real$ eine eindeutige reelle Lösung besitzt.

(Leider haben wir es auch nach viel herumprobieren nicht geschaft, \texttt{simpy} richtig entscheiden zu lassen, welche der drei Nullstellen jeweils reell sind.
Daher argumentieren wir diesen Teil rein mathematisch ohne zugehörigen Code.)

Mithilfe von scipy lassen sich die drei verschiedenen Lösungen bestimmen:

\lstinputlisting[style=pythoncode, firstline = 1, lastline = 5]{chapter_05/exercise_05_26.py}

Wir erhalten die folgenden Lösungen:

\begin{consoleoutput}
{-(-1/2 - sqrt(3)*I/2)*(-27*a/2 + sqrt(729*a**2 + 2916)/2)**(1/3)/3 + 3/((-1/2 - sqrt(3)*I/2)*(-27*a/2 + sqrt(729*a**2 + 2916)/2)**(1/3)), -(-1/2 + sqrt(3)*I/2)*(-27*a/2 + sqrt(729*a**2 + 2916)/2)**(1/3)/3 + 3/((-1/2 + sqrt(3)*I/2)*(-27*a/2 + sqrt(729*a**2 + 2916)/2)**(1/3)), -(-27*a/2 + sqrt(729*a**2 + 2916)/2)**(1/3)/3 + 3/(-27*a/2 + sqrt(729*a**2 + 2916)/2)**(1/3)}
\end{consoleoutput}

Wir können nun die reelle Lösung im geforderten Intervall $[-500, 500]$ mit dem folgenden Code plotten:

\lstinputlisting[style=pythoncode, firstline = 7, lastline = 8]{chapter_05/exercise_05_26.py}

Damit erhalten wir die folgenden Graphen:

\begin{center}
  \includegraphics[width = 0.6\textwidth]{chapter_05/exercise_05_26_figure.pdf}
\end{center}
