\section{}





\subsection{}

Wir testen, für welche Werte von $\alpha$ die Determinante von $A$ verschwindet:

\lstinputlisting[style=pythoncode, firstline = 1, lastline = 8]{chapter_05/exercise_05_24.py}

Wir erhalten den folgenden Output:

\begin{consoleoutput}
The kernel is nonzero for the following values of alpha:
{-3}
\end{consoleoutput}

Somit hat das lineare Gleichungssystem $Ax = 0$ nur für $\alpha = -3$ nicht-triviale Lösungen.





\subsection{}

Die Spalten von $A$ sind genau linear abhängig, wenn $\det A = 0$ gilt.
Nach dem vorherigen Aufgabenteil gilt dies nur für $\alpha = -3$.





\subsection{}

Für $\alpha \neq -3$ ist die Matrix invertierbar, sodass das Gleichungssystem $Ax = b$ dann für jedes $\beta \in \Real$ eine eindeutige Lösung hat.
Für $\alpha = 3$ hat die Matrix $A$ immer noch Rang $2$, weshalb sich die Werte für $\beta$ in diesem Fall wie folgt bestimmen lässt:

\lstinputlisting[style=pythoncode, firstline = 14, lastline = 17]{chapter_05/exercise_05_24.py}

Wir erhalten den folgende Output:

\begin{consoleoutput}
For alpha = -3 there exists a solution for the following values of beta:
{0}
\end{consoleoutput}





\subsection{}

Für $\alpha = -3$, $\beta = 0$ berechnen wir die Lösungen für $Ax = b$ mit \texttt{linsolve}:

\lstinputlisting[style=pythoncode, firstline = 23, lastline = 24]{chapter_05/exercise_05_24.py}

Wir erhalten den folgenden Output:

\begin{consoleoutput}
For alpha = -3, beta = 0 the solutions are given by the following set:
{(z, -2*z + 3, z)}
\end{consoleoutput}











