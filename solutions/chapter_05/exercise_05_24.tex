\section{}





\subsection{}

Wir testen, für welche Werte von $\alpha$ die Determinante von $A$ verschwindet:

\lstinputlisting[style=pythoncode, firstline = 1, lastline = 8]{chapter_05/exercise_05_24.py}

Als Ergebnis erhalten wir \texttt{\{-3\}}.
Somit hat das lineare Gleichungssystem $Ax = 0$ nur für $\alpha \neq -3$ nicht-triviale Lösungen.





\subsection{}

Die Spalten von $A$ sind genau linear abhängig, wenn $\det A = 0$ gilt.
Nach dem vorherigen Aufgabenteil gilt dies nur für $\alpha = -3$.





\subsection{}

Für $\alpha \neq -3$ ist die Matrix invertierbar, sodass das Gleichungssystem $Ax = b$ dann für jedes $\beta \in \Real$ eine eindeutige Lösung hat.
Für $\alpha = 3$ hat die Matrix $A$ immer noch Rang $2$, weshalb sich die Werte für $\beta$ in diesem Fall wie folgt bestimmen lässt:

\lstinputlisting[style=pythoncode, firstline = 10, lastline = 13]{chapter_05/exercise_05_24.py}

Wir erhalten den Wert $\beta = 0$.





\subsection{}

Für $\alpha = -3$, $\beta = 0$ berechnen wir die Lösungen des Gleichungssystems $Ax = b$ mit \texttt{linsolve}:

\lstinputlisting[style=pythoncode, firstline = 15, lastline = 16]{chapter_05/exercise_05_24.py}

Als Lösung erhalten wir \texttt{\{(z, -2*z + 3, z)\}}.










