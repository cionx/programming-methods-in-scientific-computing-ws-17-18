\section{}

Wir bestimmen die Skalarprodukte $\langle p_i, p_j \rangle$ und tragen diese in eine Matrix ein:

\lstinputlisting[style=pythoncode, firstline = 1, lastline = 10]{chapter_05/exercise_05_31.py}

Wir erhalten die folgende Matrix:

\begin{consoleoutput}
Matrix([[1, 0, 0], [0, 1/12, 0], [0, 0, 1/180]])
\end{consoleoutput}

Dies ist eine Diagonalmatrix, was die paarweise Orthogonalität der $p_i$ zeigt.
