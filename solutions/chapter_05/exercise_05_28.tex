\section{}

Wir nutzen den folgenden Code: 

\lstinputlisting[style=pythoncode, firstline = 1, lastline = 31]{chapter_05/exercise_05_28.py}

Wir testen unser Programm anhand der gegebenen Funktion:

\lstinputlisting[style=pythoncode, firstline = 35, lastline = 41]{chapter_05/exercise_05_28.py}

Wir erhalten in der Konsole den folgenden Output:

\begin{consoleoutput}
initial value                   root
Matrix([[1], [1], [0]])         Matrix([[0.893628234476483], [0.894527010390578], [-0.0400892861591528]])
Matrix([[1], [-1], [0]])        Matrix([[0.893628234476483], [-0.894527010390578], [-0.0400892861591528]])
Matrix([[-1], [1], [0]])        Matrix([[-0.893628234476483], [0.894527010390578], [-0.0400892861591528]])
Matrix([[-1], [-1], [0]])       Matrix([[-0.893628234476483], [-0.894527010390578], [-0.0400892861591528]])
\end{consoleoutput}
