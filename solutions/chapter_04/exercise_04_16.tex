\section{}

Wir passen zunächst die bisherige \texttt{newton}-Methode an, um mit scipy \texttt{array} zu arbeiten:

\lstinputlisting[style=pythoncode, firstline = 1, lastline = 18]{chapter_04/exercise_04_16.py}

Anschließend bestimmen wir die gesuchte Nullstelle:

\lstinputlisting[style=pythoncode, firstline = 22, lastline = 46]{chapter_04/exercise_04_16.py}

Wir erhalten den folgenden Output:

\begin{consoleoutput}
$ python exercise_04_16.py
initial value   root
(1, 1, 0)       [ 0.89362823  0.89452701 -0.04008929].
(1, -1, 0)      [ 0.89362823 -0.89452701 -0.04008929].
(-1, 1, 0)      [-0.89362823  0.89452701 -0.04008929].
(-1, -1, 0)     [-0.89362823 -0.89452701 -0.04008929].
\end{consoleoutput}
