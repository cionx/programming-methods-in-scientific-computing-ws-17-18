\section{}

Wir nutzen die folgende Funktion, um eine gegebene quadratische Matrix $A$ als $A = L + D + U$ wie in der Aufgabenstellung zu zerlegen:

\lstinputlisting[style=pythoncode, firstline = 4, lastline = 16]{chapter_04/exercise_04_18.py}

Wir bestimmen nun zunächst die exakte Lösung mithilfe des folgenden Codes:

\lstinputlisting[style=pythoncode, linerange={18-24, 32-35}]{chapter_04/exercise_04_18.py}

Anschließend berechnen wir mithilfe des Gauß-Seidel-Algorithmus eine approximative Lösung:

\lstinputlisting[style=pythoncode, firstline = 41, lastline = 50]{chapter_04/exercise_04_18.py}

Wir erhalten den folgenden Output:

\begin{consoleoutput}
A:
[[ 4  3  0]
 [ 3  4 -1]
 [ 0 -1  4]]
b:
[ 24  30 -24]
The solution to Ax = b is:
[ 3.  4. -5.]
An approximate solution to Ax = b is:
[ 2.57885742  4.35095215 -4.91226196]
The difference is:
[ 0.42114258 -0.35095215 -0.08773804]
\end{consoleoutput}
