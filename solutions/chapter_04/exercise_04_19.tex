\section{}

Wir interpolieren die gegebenen Werte mit einem Polynom $p$ vom Grad $7$.
Hierfür nutzen wir die Funktion \texttt{KroghInterpolator} aus dem Paket \texttt{scipy.interpolate};
mit dieser lassen sich die Ableitungen $p^{(n)}(0)$ bestimmen, aus denen sich dann die Koeffizienten bestimmen lassen:

\lstinputlisting[style=pythoncode, firstline = 1, lastline = 17]{chapter_04/exercise_04_19.py}

Wir bestimmen die Koeffizienten anschließend noch einmal durch ein entsprechendes lineares Gleichungssystem:

\lstinputlisting[style=pythoncode, firstline = 21, lastline = 28]{chapter_04/exercise_04_19.py}

Wir erhalten den folgenden Output:

\begin{consoleoutput}
$ python exercise_04_19.py
The coefficients (via interpolation) are {}:
[1.0, -13.666666666666661, 65.46666666666664, -110.66666666666664, 81.333333333333314, -26.666666666666661, 3.1999999999999993]
The coefficients (via linear equations) are {}:
[   1.          -13.66666667   65.46666667 -110.66666667   81.33333333
-26.66666667    3.2       ]
\end{consoleoutput}
