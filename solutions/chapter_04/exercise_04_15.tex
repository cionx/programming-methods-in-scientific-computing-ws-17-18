\section{}



\subsection{}

Wir plotten zunächst die beiden Funktionen:

\lstinputlisting[style=pythoncode, linerange={1-3, 5-13}]{chapter_04/exercise_04_15.py}

Wir erhalten den folgenden Graphen:

\begin{center}
  \includegraphics[width = 0.6\textwidth]{chapter_04/exercise_04_15_figure.pdf}
\end{center}

Wir bestimmen nun die beiden Nullstellen mit dem folgenden Code:

\lstinputlisting[style=pythoncode, firstline = 15, lastline = 20]{chapter_04/exercise_04_15.py}

Wir erhalten den folgenden Output:

\begin{consoleoutput}
The intersections are at 0.5893877634693506 and 2.1267998926782568.
\end{consoleoutput}



\subsection{}

Die gegebene Funktion $f$ ist konvex;
für einen beliebigen Startwert $x_0$ mit $f'(x_0) \neq 0$ konvergiert daher das Newton-Verfahren gegen eine der beiden Nullstellen.
Der einzige kritische Punkt ist daher der eindeutige Wert $x \in \Real$ mit $f'(x) = 0$.
Wir bestimmen diesen Wert näherungsweise:

\lstinputlisting[style=pythoncode, firstline = 24, lastline = 25]{chapter_04/exercise_04_15.py}

Wir erhalten das folgende Ergebnis:

\begin{consoleoutput}
The newton method cannot start at 1.4436349751811388.
\end{consoleoutput}
